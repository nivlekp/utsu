\documentclass{article}
\usepackage[T1]{fontenc}
\usepackage[a4paper, landscape]{geometry}
\usepackage{fontspec, graphicx, nopageno, tikz}
\usepackage{xeCJK}
\setCJKmainfont{MS Mincho}

\parindent=0pt
\parskip=12pt

\begin{document}
Every day is a struggle.
\clearpage
\section*{Performance notes}
{\bfseries Accidentals}: Accidentals only apply to the note they
immediately precede, except in the case of a tie, where the accidental
applies to all the notes in the tie.

{\bfseries Timing}: While there should not be any intended rubato, the
composer is aware that some of the rhythms are challenging to the
performer. The performer should try their best without feeling too
stressed.

{\bfseries Playability}: Although an ``average" person's handspan was
taken into consideration, there are instances where it might be
difficult to play two far-apart notes simultaneously. There are two
cases to consider:
\begin{enumerate}
\item {\bfseries Two notes starting at the same time}: The performer
shall treat one of them as a grace note. The choice is left for the
performer.
\item {\bfseries One note starting while the other one is being played}:
The performer shall let go of the ``old" note and start playing the
``new" note.
\end{enumerate}
\end{document}
